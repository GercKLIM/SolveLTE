\documentclass[12pt, a4paper]{article}

\usepackage[utf8]{inputenc}
\usepackage[T1]{fontenc}
\usepackage[russian]{babel}
\usepackage[oglav,spisok,boldsect,eqwhole,figwhole,hyperref,hyperprint,remarks,greekit]{./style/fn2kursstyle}
\graphicspath{{./style/}{./figures/}}
\usepackage{caption}
\usepackage{subcaption}

\usepackage{multirow}
\usepackage{supertabular}
\usepackage{multicol}



\usepackage{listings}
\usepackage{color}

\definecolor{dkgreen}{rgb}{0,0.6,0}
\definecolor{gray}{rgb}{0.5,0.5,0.5}
\definecolor{mauve}{rgb}{0.58,0,0.82}

\lstset{frame=tb,
	inputencoding=utf8,
	basicstyle=\ttfamily,
	extendedchars=\true,
	showspaces=\false,
	showstringspaces=\false,
	language=Python,
	aboveskip=3mm,
	belowskip=3mm,
	showstringspaces=false,
	columns=flexible,
	basicstyle={\small},
	numbers=left,
	numberstyle=\tiny\color{gray},
	keywordstyle=\color{red},
	commentstyle=\color{dkgreen},
	stringstyle=\color{mauve},
	breaklines=true,
	breakatwhitespace=true,
	tabsize=3
}











\title{Распознование графиков решения \\ одномерного линейного уравнения переноса}
\author{О.\,Д.~Климов}
\supervisor{М.\,П.~Галанин}
\group{ФН2-61Б}
\date{2024}

\renewcommand{\labelenumi}{\theenumi)}
\begin{document}

\maketitle

\tableofcontents

\newpage

\section-{Введение}
Необходимость распознавания графика функций возникает в совершенно разных прикладных задачах науки и техники. Например, она непосредственно связана с проблемой восстановления графика решений уравнений по неточно заданным данных о решениях. В силу нелинейности распознавания изображений нахождение точных алгоритмов для такой задачи испытывало ряд трудностей. Однако с развитием программирования и вычислительной техники стало возможным решать данную задачу методами нейронных сетей.

\section{Постановка задачи}

\subsection{Формулировка}
Необходимо изучить методы численного решения линейного одномерного уравнения переноса. Составить и отладить программу для нахождения численного решения задачи Коши для указанного уравнения. Использовать шесть различных разностных схем:
\begin{enumerate}
	\item Явную схема с левой разностью на двух точках
	\item Явную схема с левой разностью на трех точках
	\item Неявную схема с левой разностью на двух точках
	\item Неявную схема с левой разностью на трех точках
	\item схема Лакса
	\item схема Лакса-Вендрофа
\end{enumerate}
Для всех схем использовать одинаковую систему тестов:
\begin{enumerate}
	\item Левый треугольник
	\item Правый треугольник
	\item Прямоугольник
	\item Синус
	\item "Зуб"
\end{enumerate}
Реализовать модель нейронной сети для распознавания решения на языке программирования Python. На основе модели разработать программу, которая по неточному решению возвращает более точное известное решение.

\subsection{Пример}
Основная задача работы состоит в том, чтобы создать программу, которая получала бы на вход неточное решение уравнения переноса и выдавала более точное соответствующее решение.


\section{Задача Коши для линейного одномерного уравнения переноса}
Уравнение переноса является одним из фундаментальных уравнений математической физики, которое широко используется для описания движения сплошной среды. Рассмотрим задачу Коши для уравнения переноса следующего вида:
\begin{equation}
	\begin{cases}
		\dfrac{\partial u}{\partial t} + \dfrac{\partial (a u)}{\partial x} = 0, \\
		u(x, 0) = u_0(x),
	\end{cases} 
	\text{где } u = u(x, t), \quad a = const > 0, \quad t \in (0, T), \quad x \in (-\infty, +\infty) .
\end{equation}


Приведем аналитическое решение. Уравнение можно записать в виде:
\begin{equation*}
	u_t + a u_x = 0
\end{equation*}
Запишем и решим характеристическое уравнение:
\begin{equation*}
	\dfrac{dt}{1} = \dfrac{dx}{a} = \dfrac{du}{0} \quad \implies  \quad 
	\begin{cases}
		u = C_1,\\
		x = a t + C_2
	\end{cases} 
	\quad \implies \quad
	\begin{cases}
		C_1 = u,\\
		C_2 = x - at
	\end{cases}
	 \quad \implies \quad
	 u = \psi(x-a t).
\end{equation*}
Применим граничные условия:
\begin{equation*}
	u(x, 0) = u_0(x),
	\quad \implies \quad
	\psi(x) = u_0(x),
	 \quad \implies \quad
	 u = u_0(x-a t).
\end{equation*}
Получим аналитическое решение уравнения $u = u_0(x-at)$.

\section{Численные методы решения задачи}
\subsection{Описание методов}
Для численного решения задачи Коши для линейного одномерного уравнения переноса рассмотрим следующие схемы:

\begin{enumerate}
	
	\item \bf{Явная схема с левой разностью по двум точкам}:
	\begin{equation*}
		\widehat{y} = (1 - \gamma) y + \gamma y_{-1}.
	\end{equation*}
	
	\item \bf{Неявная схема с левой разностью по двум точкам}:
	\begin{equation*}
		\widehat{y} = \dfrac{\gamma}{1 + \gamma} \widehat{y}_{-1} + \dfrac{1}{1 + \gamma} y .
	\end{equation*}
	
	\item \bf{Явная схема с левой разностью по трем точкам}:
	\begin{equation*}
		\widehat{y} = (1 - \frac{3}{2}\gamma) y + \gamma(2y_{-1} - \frac{1}{2}y_{-2}) .
	\end{equation*}
	
	\item \bf{Неявная схема с левой разностью по трем точкам}:
	\begin{equation*}
		%\widehat{y} = (1 - \frac{3}{2}\gamma) y + \gamma(2y_{-1} - \frac{1}{2}y_{-2}).
	\end{equation*}
	
	\item \bf{Схема Лакса}:
	\begin{equation*}
		\widehat{y} = \dfrac{(y_{+1} + y_{-1}) - \gamma(y_{+1} - y_{-1})}{2} .
	\end{equation*}
	
	\item \bf{Схема Лакса-Вендрофа}:
	\begin{equation*}
		\widehat{y} = y - \gamma(F_p - F_l),
	\end{equation*}
	где $ F_p = \dfrac{(y_{+1} + y) - \gamma (y_{+1} - y)}{2}, \quad F_l = \dfrac{(y + y_{-1}) - \gamma (y - y_{-1})}{2} $.
\end{enumerate}

\subsection{Реализация}
\subsection{Тестирование}


\section{Метод улучшения решения с помощью нейронной сети}
\subsection{Модель сверточной нейронной сети}
\subsection{Распознавание решения}
\subsection{Результаты работы программы}


\section{Актуальность и перспективы задачи}
\section-{Заключение}



\clearpage
\begin{thebibliography}{1}

\bibitem{1} Галанин М.П., Савенков Е.Б. Методы численного анализа математических моделей. М.: Изд-во МГТУ им. Н.Э. Баумана. 2018. 592 с.


\bibitem{2} Tariq Rashid Make Your Own Neural Network // CreateSpace Independent Publishing Platform; 1st edition SAND96-0583 (March 31, 2016)


%\bibitem{4} TensorFlow documentation. URL: https://www.tensorflow.org/?hl=ru (Дата обращения 17.01.2024)

%\bibitem{5} Keras documentation URL: https://keras.io (Дата обращения 17.01.2024)

%\bibitem{6} Гафаров Ф.М. Искусственные нейронные сети и приложения. г.Казань: Изд-во Казан. ун-та, 2018. –121 с


\end{thebibliography}

\end{document} 